\documentclass{article}
\usepackage[utf8]{inputenc}

\title{Домашнее задание № 4}
\author{Иван Нечепуренко }
\date{September 2018}

\usepackage{natbib}
\usepackage{graphicx}
\usepackage{amsmath,amsfonts,amssymb,amsthm,mathtools} % AMS
\usepackage[english,russian]{babel}	% локализация и переносы

\begin{document}

\maketitle

\section{Задача 1}
Для начала пронумеруем все условия эксремума из теоремы Каруша-Куна-Такера, чтобы затем быстро на них ссылаться:
\begin{align*}
&1) g_i(x^ *) = 0, i = 1, . . . , m \\
&2) h_j(x^*) \leq 0, j = 1, . . . , p \\
&3) \mu^*_j \geq 0, j = 1, . . . , p \\
&4) \mu^*_j h_j(x^*) = 0, j = 1, . . . , p \\
&5) \Delta_x L(x^*, \lambda^*, \mu^*) = 0 
\end{align*}

Запишем Лагранджиан:
$$ L(x, \lambda, \mu) = (x - 3)^2 + (y - 2)^2 + \lambda (y - x -1) + \mu (y + x - 3)$$

Применяем 5-е условие, получаем:

$$2(x - 3) - \lambda + \mu = 0$$ 
$$ 2(y - 2) + \lambda + \mu = 0$$

Подставим 1-е условие ($y = x + 1$), имеем:

$$2(x - 3) - \lambda + \mu = 0$$ 
$$ 2(x - 1) + \lambda + \mu = 0$$

Сложим эти два равенства, получим 

$$2(x - 2) + \mu = 0 $$

Ещё запишем условие 4), подставив $y = x + 1$, имеем:

$$\mu (x - 1) = 0$$. 

Совместив последне два равенства, получаем возможные решения $x = 1, \mu = 2$ и 
$x = 2, \mu = 0$. Оба случая подходят под 3-е условие, а значит, остается рассмотреть точки $(1, 2)$ и $(2, 3)$. Но в любом случае, гессиан нашей функци - $diag(2, -2)$. Это значит, что вторая однозначно являеся седловой, но в то же время первая леж на границе,  про неё рассуждать не так уж просто. На самом деле, первая точка-действитеьно точка минимума: можно решить "в лоб", можно заметить, что из-за условий огранчения на векор $d(x, y)$ наложно условие $dy > dx$, и при таком условии гессиан отрицательно определен. Ну, еще возможный способ - выразить $y$ через $x$ и тогда уже получить выпуклую задачу, но это, кажется, не совсем то, что требуется

\section{Задача 2}
Перепишем это уравнение в стандартном виде:

\begin{align*}
min \,\,\,\, x_1^2 + 2x_2^2 + &x_3 \\
x_1 - 2x_2 + 3x_3 - 4 &\leq 0 \\
-x1 + 2x2 - 3x3 - 4 &\leq 0
\end{align*}

Записываем 5-е условие:
\begin{align*}
2x_1 + (\mu_1 - \mu_2) &= 0 \\
4x_2 - 2(\mu_1 - \mu_2) &= 0 \\
1  + 3(\mu_1 - \mu_2) &= 0
\end{align*}
Получается: $\displaystyle \mu_1 - \mu_2 = -\frac{1}{3}$, $\displaystyle x_1 = \frac{1}{6}$, $\displaystyle x_2 = -\frac{1}{6}$.
Если подставить это в условия типа 4, получим:
\begin{align*}
\mu_1(18x_3 - 21)&=0 \\
\mu_2(18x_3 + 27)&=0 \\
\end{align*}
Возможные значения $x_3$: $\displaystyle \frac{7}{6}$ и $\displaystyle -\frac{3}{2}$, иначе $\mu_1 = \mu_2 = \mu_1 - \mu_2 = 0$. Но если $\mu_2 = 0$, то $\displaystyle \mu_1 = -\frac{1}{3}$, а это противоречит условию типа 3). Единственная точка эксттремума - $\displaystyle x = -\frac{3}{2}$, на ней функия достигает наименьшее значение: $\displaystyle -\frac{13}{12}$ (гессан оптимзруемой функции - $diag(2, 4, 0)$, ограничения линейны, можно пользоваться, как минимум, выпуклостью задачи).
\section{Задача 3}
Задача сводится к стандартному виду довольно тривиально:$\\
min (x_1 - 2)^2 + (x_2 + 1)^2  \\
x_1 + x_2 + 2 \leq 0 \\
x_1 - x_2 \leq 0 \\
$
Как всегда, наинаем с 5 условия:$ \\
2(x_1 - 2) + \mu_1 + \mu_2 = 0 \\
2(x_2 + 1) + \mu_1 - \mu_2 = 0 \\
$
Вычтя из первого условия второе, и прибавив к нему второе, получаем эквивалентную систему: $ \\
x_1 = x_2 + 3 -\mu_2 \\
x_1 = -x_2 + 1 -\mu_1 \\
$
Осается только делать перебор, полагая равенствонеравенство $\mu_i$ нулю, и рассматривая условия типа 4). Опустив однотипные выкладки, получим следующие кортежи $(x_1, x_2, \mu1 , \mu_2)$: $(2, -1, 0, 0)$, $(0.5, 0.5, 0, 3)$, $(0.5, -2.5, 3, 0)$, $(-1, -1, 3, 3)$. Но под условие типа 2) подходит только точка $(-1, -1, 3, 3)$. Гессиан функции - $diag(2, 2)$, ограничения линейны, а это значит, что задача - выпуклая, наша точка - точка глобального минимума, и значение на ней - $9$
\section{Задача 4}
Стандартный вид: 
\begin{align*}
 min -3x_1^2 + x_2^2 + 2x_3^2 + 2(x_1 &+ x_2 + x_3) \\
 s. t. x_1^2 + x_2^2 + x_3^3 - 1 &= 0
\end{align*}
Записываем 5-е условие:
\begin{align*}
-6x_1 + 2 + 2 \lambda x_1 = 0 \\
2x_2 + 2 + 2 \lambda x_2 = 0 \\
4x_3 + 2 +2\lambda x_3 = 0 
\end{align*}
Перепаисываем в более удобной форме (деление оправдано, исключительных случаев нет): $$x_1 = -\frac{1}{-3 + \lambda} $$
$$ x_2 = -\frac{1}{1 + \lambda} $$
$$ x_3 = -\frac{1}{2 + \lambda} $$
Подставляя выражение в условие $ x_1^2 + x_2^2 + x_3^2 = 1$, мы получаем методом численного решения следующие допустимые $\lambda$: $3.14929, 0.223509, 1.8919, 4.03523$. Уже сейчас мы можем посчитать соответствующие данным значениям:
$\\-120.20390447965981$, $2.4971731870372755$, $-2.7910274873038787$, $\\-0.04441007738681746$
соответственно. Отсюда очевидно, что $(6.698372295532179, \\ 0.24100508761739958, 0.19420153069646492)$ - точка глобального минимума, $-120.20390447965981$ - сам минимум.

Минимализируемая функция - невыпукла (градиент целевой функции - $diag(-6, 2, 4)$, да и в выпуклой вряд ли легко сводится - есть четыре критические точки.
\end{document}