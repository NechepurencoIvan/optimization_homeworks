\documentclass[tikz]{article}
\usepackage[utf8]{inputenc}

\title{Домашнее задание № 5}
\author{Иван Нечепуренко }
\date{September 2018}

\usepackage{natbib}
\usepackage{graphicx}
\usepackage{amsmath,amsfonts,amssymb,amsthm,mathtools} % AMS
\usepackage[english,russian]{babel}	% локализация и переносы

%%% Работа с картинками
\usepackage{graphicx}  % Для вставки рисунков
\graphicspath{{images/}}  % папки с картинками
\setlength\fboxsep{3pt} % Отступ рамки \fbox{} от рисунка
\setlength\fboxrule{1pt} % Толщина линий рамки \fbox{}
\usepackage{wrapfig} % Обтекание рисунков и таблиц текстом

\begin{document}

\section{Задача 1}

2) Для начала сделаем замену:
$$ x_i = \sigma_{i+} - \sigma_{i-}$$
$$\sigma_{i+},  \sigma_{i-} \geq 0$$
Заметим, что таким образом задаются можно задать любое число. Более, таких способов бесконечно много. Но нас интересуют конкретные  возможные значения выражения $\sigma_{i+}+
\sigma_{i-}$. Оно явно не меньше нуля, но мы утверждаем, что эта сумма не больше модуля. Действительно, модуль достигается в случае хотя бы одного из слагаемых, равного $0$, в зависимости от знака $x_i$. Но, если прибавлять к одной из сигм некоторое число, придется прибавить его другой, а это приведет к увеличению $\sigma_{i+} + \sigma{i-}$. Но вычесть число мы не можем, т. к. одно из слагаемых равно 0. Получаем алгоритм приведения к классическому виду: выполняем замену, сделанную в начале, в целевой функции модуль меняем на 
$\sigma_{i+} + \sigma{i-}$. Мы можем взять $x_i$ из решения оригинальной задачи, подобрать 
$\sigma{i+-}$, так, что по одной из них будет $0$, в целевой функции снова появятся модули.
В обратную сторону док-во тоже прогоняется. Но это еще не все, ведь нам нужно получить каноническую форму. Все стандартно: добавляем по одной переменной на неравенство, и они будут не меньше 0.

1) Мы научились избавляться от модулей в целевой функции. Но сначала преобразуем ограничения.
Мы можем заметить, что ограничение на сумму модулей можно заменить совокупностью ограничений,
где каждый модуль раскрывается со знаком $+$ $/$ $-$. Для каждой совокупности $x_i$ одно 
из таких ограничений будет ограничением на модули. С другой сторны, если выполнено ограничение с модулями, то выполнены все остальные.  Все, можно сделать замену $x_2' = x_2 - 10$, чтобы совсем по алгоритму делать. Получается:
$$\min 2(\sigma_{1+} + \sigma_{1-}) + 3(\sigma'_{2+} + \sigma'_{2-})$$
$$ (\sigma_{1+} - \sigma_{1-} + 2) + (\sigma_{2+} - \sigma_{2-} + 10) = -x_3$$
$$ - (\sigma_{1+} - \sigma_{1-} + 2) + (\sigma_{2+} - \sigma_{2-} + 10) = -x_4$$
$$ (\sigma_{1+} - \sigma_{1-} + 2) - (\sigma_{2+} - \sigma_{2-} + 10) = -x_5$$
$$ - (\sigma_{1+} - \sigma_{1-} + 2) - (\sigma_{2+} - \sigma_{2-} + 10)  = -x_6$$
Ну, и все переменные неотрицательны.

3) Легче рассмотреть данное нам по строчкам. Пусть $a_i$ - строки матрицы $A$. Пусть
точка $x_c + r$ принадлежит сфере с радиусом $R$. Посмотрим, как для неё выполняется условие
принадлежности множеству. Можно записать условие того, что вся сфера в полиедральном множестве:
$$a_i (x_c + r) \leq b_i \forall r ||r||_2 \leq R$$
$$\sup\limits_r\{a_i (x_c + r): ||r||_2 \leq R\} \leq b_i$$
Можно из минимальных соображений линейной алгебры получить значение супремума:
$$a_i x_c + R||a_i||^2 \leq b_i$$
Мы сводим задачу к линейному программированию:
$$ \min -R$$
$$ Ax_c \leq -R||a'|| + b$$
$||a'||$ состоит из чисел $||a_i||_2$. Остается сделать приведение к каноническому виду:
$$ \min -R$$
$$ Ax_{c+} + Ax_{c-}= -R||a'|| + b - y$$
Все переменные неотрицательны.
\section{Задача 2}
nb=
ти

Для начала приводим задачу к каноническому виду. Для этого Вводим дополнительные перемненные.
В принципе, это все, что от нас требуется. Получаем:
$$ min -5x_1 - x_2$$
$$ x_1 + x_2 + x_3 = 5$$
$$ 2x_1 + 0.5x_2 + x_4 = 8$$
$$ x_1, .. , x_4 \geq 0 $$

ииии

\begin{tabular}{|l|l|l|l|l|l|l|}
\hline
Базис & H & $x_1$ & $x_2$ & $x_3$ & $x_4$ & b \\
$x_3$ & 5 & 1     & 1     & 1     & 0     &  \\
$x_4$ & 8 & 2     & 0.5   & 0     & 1     &  \\
c     &   & -5    & -1    & 0     & 0     &  \\
\hline
\end{tabular}

Можно увидеть, что максимальное по модулю значение в столбце $c$ дотигается при колонке $x_1$. Можно теперь посчитать значение столбца $b$, равное поэлементному частному $H$ и 
$x_1$. 

\begin{tabular}{|l|l|l|l|l|l|l|}
\hline
Базис & H & $x_1$ & $x_2$ & $x_3$ & $x_4$ & b \\
$x_3$ & 5 & 1     & 1     & 1     & 0     &  5\\
$x_4$ & 8 & 2     & 0.5   & 0     & 1     &  4\\
c     &   & -5    & -1    & 0     & 0     &  \\
\hline
\end{tabular}

Наименьшее значение в последнем столбце соответствует $x_4$. В пересечении нужных строки и столбца лежит элемент $2$. Все, что нам остается, это поменятб базис. Получаем:

\begin{tabular}{|l|l|l|l|l|l|l|}
\hline
Базис & H & $x_1$ & $x_2$ & $x_3$ & $x_4$ & b \\
$x_3$ & 1 & 0     & 0.75     & 1     & -0.5     &  3\\
$x_1$ & 4 & 1     & 0.25   & 0     & 0.5     &  2\\
c     & 20 & 0 & 0.25  & 0     & 0     &  \\
\hline
\end{tabular}

Мы видим, что внизу остаются только положительные числа, а это значит, что наш метод сошелся за один шаг.  Получается ответ: $x_1 = 4$, $x_3 = 1$, отсюда $x_2 = 0$. 
График схождимости приложен отдельно


\end{document}