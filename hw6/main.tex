\documentclass{article}
\usepackage[utf8]{inputenc}

\title{Домашнее задание № 5}
\author{Иван Нечепуренко }
\date{September 2018}

\usepackage{natbib}
\usepackage{graphicx}
\usepackage{amsmath,amsfonts,amssymb,amsthm,mathtools} % AMS
\usepackage[english,russian]{babel}	% локализация и переносы

%%% Работа с картинками
\usepackage{graphicx}  % Для вставки рисунков
\graphicspath{{images/}}  % папки с картинками
\setlength\fboxsep{3pt} % Отступ рамки \fbox{} от рисунка
\setlength\fboxrule{1pt} % Толщина линий рамки \fbox{}
\usepackage{wrapfig} % Обтекание рисунков и таблиц текстом

\begin{document}


- Для функции Розенброка $f(x_1, x_2) = 100 (x_2 - x^2_1)^2 + (1 - x_1)^2$ найдите аналитически её минимум и исследуйте зависимость сходимости градиентного спуска от начальных приближений $x^0 = (1.2, 1.2)$ и $x^0 = (-1.2, 1)$. 
Величину шага определяйте только по условию достаточного убывания функции. 
Также постройте график зависимости длины шага от итерации для каждой начальной точки. Какой вывод Вы можете сделать?


$$ \frac{\partial f}{\partial x_2} = 200(x_2 - x_1^2)  = 0 $$
$$ \frac{\partial f}{\partial x_1} = 400(x_2 - x_1^2)x_1 + 2(1 - x_1) = 0$$
Подставив первое уравнение в первое, мы получим, что $2(1 - x_1) = 0$, а значит, $x_1 = 1$
Но возврацаясь к первому уравнению, мы можем получить $x_2 = 1$. Мы получаем, что $(1, 1)$ - еденственная точка, в которой градиент обращается в $0$. Заметим, что при устремлении $x_1$,
$x_2$ к бесконечности, функция неограниченно возрастает, мы мажем прийти к выводу, что $(1, 1)$ - единственная точка минимума.

- Решите задачу наискорейшим спуском
$$
\frac{1}{2}x^{\top}Ax - b^{\top}x \to \min_x
$$
при 
$$
A = 
\begin{bmatrix}
0.78 & −0.02 & −0.12 & −0.14\\
−0.02 & 0.86 & −0.04 & 0.06 \\
−0.12 & −0.04 & 0.72 & −0.08\\
−0.14 & 0.06 & −0.08 & 0.74
\end{bmatrix}
\qquad
b = \begin{bmatrix}
0.76\\
0.08\\
1.12\\
0.68
\end{bmatrix}
$$
и начальной точкой $x^0 = 0$ с точностью нормы градиента $10^{-6}$. Постройте график сходимости, насколько сходимость быстрая? Найдите спектр матрицы $A$ (покажите, как Вы его искали) и сравните полученный график сходимости с графиком сходимости, полученным из теоретических соображений.

Найдем шаг градиентного спуска,  соответстыующий наискорейшему спуску. Запишем
миннимизацию для $(x - \alpha f'(x))$
$$\frac{1}{2}(x - \alpha f'(x))^TA(x - \alpha f'(x))  - b^T(x - \alpha f'(x)) \to \min_\alpha$$
$$\frac{1}{2} x^TAx - b^Tx  - \alpha x^T  A f'(x) + \frac{1}{2}\alpha^2f'(x)^T A f'(x) + \alpha b^T f'(x)
\to \min_\alpha$$

$$\frac{1}{2}\alpha^2f'(x)^T A f'(x) + \alpha ( b^T f'(x) - x^T  A f'(x))
\to \min_\alpha$$

Это - квадратный трехчлен с неотрицательным главным членом, а значит, записываем 
$$ \alpha = \frac{-b}{2a} = \frac{x^TAf'(x) - b^Tf'(x) }{f'(x)^TAf'(x)} = 
\frac{f'(x)^Tf'(x)}{f'(x)^TAf'(x)}
$$

Покажите, что градиенты, полученные на двух последовательных шагах градиентного спуска с использованием правила наискорейшего спуска, ортогональны. 

Введем следующие обозначения: $x_k$ точка, из которой делается первый шаг.
Заметим, что нам, фактически, требуется доказать, что $(\Delta f(x_k), \Delta f(x_{k+1}) = 0$. Рассмотрим сужение функции на прямую, содержащую $x_k$, и параллельную градиенту в ней.
Просто из определения минимального спуска следует, что производная этой функции в точке $x_{k + 1}$(т. е. в соответствующем ей параметре) равна нулю. Но из математического анализа
знаем, что эта производная и есть проекция градиента в точке на прямую. А отсюда сразу следует ортогональность.

Докажите, что если целевая функция $f(x) = \frac{1}{2}x^{\top}Qx - b^{\top}x$, $Q \in \mathbb{S}^n_{++}$, шаг определяется с помощью наискорейшего спуска и $x^0 - x^*$ параллелен собственному вектору матрицу $Q$, то градиентный спуск сходится за один шаг. Покажите, что для $f(x) = x^{\top}x$ градиентный спуск с выбором шага по правилу наискорейшего спуска сходится за одну итерацию (с использованием предыдущего утверждения и прямым вычислением).

Немного некрасиво сделал: начал с третьего задания, а там уже часть теории используется. Наверное, следует начать проверку с него, так как будут ссылки. Представим $x = x_0 + h$,
притом $Qx_0 = b$ из  того, что в этой точке градиент обращается в 0, и $Qh = \lambda h$,
по определению собственного вектора. Но тогда распишем выражение для шага, показанное в 3
задаче: 
$$\alpha =  \frac{(Qx - b)^T(Qx - b)}{(Qx - b)^TQ(Qx - b)} = 
 \frac{(Qx_0 + Qh - b)^T(Qx_0 + Qh - b)}{(Qx_0 + Qh - b)^TQ(Qx_0 + Qh - b)}
$$ 
$$\alpha = 
\frac{(b + \lambda h - b)^T(b + \lambda h - b)}{(b + \lambda h - b)^TQ(b + \lambda h - b)} =
\frac{\lambda^2 h^T h}{\lambda^3 h^T h } = \frac{1}{\lambda}
$$
Получаем $$ x_1 = x* - \frac{1}{\lambda}(Q(x_0 + h) - b) =
x* - \frac{1}{\lambda}(b + \lambda h - b) = x* - h = x_0
$$
Что и требовалось доказать
\end{document}