\documentclass{article}
\usepackage[utf8]{inputenc}

\title{Домашнее задание № 1}
\author{Иван Нечепуренко }
\date{September 2018}

\usepackage{natbib}
\usepackage{graphicx}
\usepackage[english,russian]{babel}	% локализация и переносы

\begin{document}

\maketitle

\section{Посели меня, если сможешь}
$\; \; \; \; \;$Прежде всего, введем исходный вектор $x = (a_{1,1}, .. , a_{n, m})$, $a_{i, k} \in  \{0, 1\} $, где каждая переменная $a_{i, j} $ - индикатор того, то i-й студент поселен в k-ю комнату. Тогда можно указать минимализируемую функцию: это $-(\sum\limits_{i, k}a_{i, k}b_{i, k} + \sum\limits_{i, j, k}a_{i, k}a_{j, k}p_{i, j})$(возможно, сумма этих сумм должна быть как-то взвешена). Здесь (во второй сумме) умножение индикаторов очень удобно перерастает в логическое умножение. Далее, первое ограничение - это то, что каждый студент должен быть заселен: $\sum\limits_{k = 1}^{m}a_{i, k} = 1, i = 1 .. n$. Также вводится ограничение на колличество студентов в одной комнате: $\sum\limits_{i = 1}^{n}a_{i, k} \leq 3, k = 1 \; .. \; m$. Эти ограничения достаточны в рамках нашей задачи. Что касается недостающих данных, то самый очевидный вариант - занулить все неизвестные переменные - достаточно хорош, если все значения $p_{i, j}$ неотрицательны. В таком случае мы имеем пессимистическую оценку: решение задачи даст нам гарантированный, хотя, возможно, не самый лучший вариант. Иначе можно пробовать делать какие-то предсказания, усреднять коэффициенты (использовать машинное обучение?) - задача творческая, но очень сильно опирающаяся на необговоренную модель отношений между студентами, и не гарантирующая результат.

\section{Перевозчик}
$\; \; \; \; \;$Это - именно та задача, из-за формулировки которой могут возникуть недопонимания между заказчиком и работником. Вектор $x$ почти очевиден: $x = (x_{1,1}, .. , x_{n, m})$, n - кол-во складов, m - магазинов, $x_{i, j}$ - кол-во перевозимого из i-го склада в j-й магазин товара, скорее всего, целое число. Очевидна также пара ограничений: $\sum\limits_{i = 1}^{n}x_{i, j} = b_j, \; j = 1 .. m$, $\sum\limits_{j = 1}^{m}x_{i, j} \leq a_i, i = 1 .. n$. А вот что дальше требуется от нас, не совсем понятно. Могут быть функции для минимализации: суммарная стоимость ($\sum\limits_{i, j}x_{i, j}c_{i, j} $), максимальное время доставки ($ max(t_{i, j}*Id(x_{i,j} \neq 0))$), или некоторая взвешенная сумма двух предыдущих функций. Или, наоборот, мы имеем ограниченное число времени/денег, и тогда предыдущие функции уже выступают в роли ограничений. Необходимо уточнять задачу.

\section{Это норма!}

$\; \; \; \; \;$1) Норма в векторном пространстве $P$ есть функция $P \to R_+$, удовлетворяющая следующим условиям: \\
$||x|| = 0$ только при $x = 0$;\\
$||x+y|| \leq ||x||+||y||$ для всех $x, y \in P$ (неравенство треугольника);\\
$||\alpha\, x||=|\alpha|*||x||$ для любого скаляра $\alpha$.\\

Нормы $||x||_1, ||x||_2$ называются эквивалентными, если 
$\exists c_1 > 0, c_2 > 0$, $\forall x \; c_1||x||_1 \leq ||x||_2 \leq c_2||x||_1 $.
Если переписать это соотношение в виде 
$c_1||x||_1 \leq ||x||_2, $
$c_2||x||_1 \geq ||x||_2 $, то становится довольно очевидно утверждение, что мы имеем дело действительно с отношением эквивалентности. 

Положим $x = (x_1, .. , x_n)$. Тогда: \\
$l_1(x) = |x_1| + .. |x_n|$\\
$l_2(x) = \sqrt{(x_1)^2 + .. (x_n)^2}$\\
$l_\infty(x) = max_{i = 1 .. n}(|x_i|)$\\

Из утверждения об эквивалентности нам достаточно доказать только эквивалентности
первых двух норм третьей. Для $l_1$ и $l_\infty$:\\
$l_\infty(x) = max_{i = 1 .. n}(|x_i|) \leq |x_1| + .. |x_n| = l_1(x) \leq
n*max_{i = 1 .. n}(|x_i|) = n*l_\infty(x)$

Для $l_2$ и $l_\infty$ имеем:\\
$l_\infty(x) = max_{i = 1 .. n}(|x_i|) = \sqrt{max_{i = 1 .. n}(|x_i|)^2} \leq 
\sqrt{(x_1)^2 + .. (x_n)^2} = l_2(x) = \sqrt{(x_1 + .. + x_n)^2} \leq 
\sqrt{(n*max_{i = 1 .. n}(|x_i|))^2} = n*max_{i = 1 .. n}(|x_i|) = n*l_\infty(x)$\\
Эквивалентность этих норм доказана.

2) Норма матрицы $A$, порожденная векторной нормой задается выраженикем $\sup_{x \neq 0}\frac{||Ax||}{||x||}$.). Из элементарных соображений линейной алгебры следует, что нам достаточно рассматривать только векторы с единичной нормой(нормирование не меняет значения $\frac{||Ax||}{||x||}$ ).\\ 

Рассмотрим $L_{\inf}$. По определению,  $\exists x, ||x|| = 1$
$L_{\inf}(A) = \max\limits_{1\leq i \leq m } \left| \sum\limits_{j = 1}^{n}a_{i, j}x_j\right|  \leq 
\max\limits_{1\leq i \leq m } \left| \sum\limits_{j = 1}^{n}a_{i, j}\right|$
($ |x_j| \leq ||x|| = 1$), а т. к. мы всегда можем взять $|x_j| = 1, sign(x_j) = sign(a_{i, j})$, то подстановкой уюеждаемся, что верхнее ограничение можно достигнуть, и 
$L_{\inf}(A) = \max\limits_{1\leq i \leq m } \left| \sum\limits_{j = 1}^{n}a_{i, j}\right|$.

Теперь перейдем к $L_1$.  $\exists x, ||x|| = 1, L_1(A) = ||Ax||_1$
$.
L_{1}(A) = \sum\limits_{i = 1}^{m} \left| \sum\limits_{j = 1}^{n} a_{i, j}x_j \right|
\leq
\sum\limits_{i = 1}^{m} \sum\limits_{j = 1}^{n} |a_{i, j}| |x_j|
=
\sum\limits_{j = 1}^{n} |x_j| \sum\limits_{i = 1}^{m} |a_{i, j}|
\leq
||x||_1 \max\limits_{j = 1 .. n} \sum\limits_{i = 1}^{m} |a_{i, j}|
=
\max\limits_{j = 1 .. n} \sum\limits_{i = 1}^{m}  |a_{i, j}|
$ \\

Но с другой стороны, нетрудно понять, что данная оценка достигается на векторе 
$x: x_j = 1, x_i = 0, i \neq j, j = \arg(\max\limits_{j = 1 .. n} \sum\limits_{i = 1}^{m} |a_{i, j}|)$. Значит, мы имеем $L_{1}(A) = \max\limits_{j = 1 .. n} \sum\limits_{i = 1}^{m}  |a_{i, j}|$

Перейдем же, наконец, к $L_2$. Для этого воспользуемся сингулярным разложением: $A = UJV^{T}$, $UU^{T} = E, VV^{T} = E, J = diag(\sigma_1, .. , \sigma_n)$. Иными словами, матрицы  $U, V$ - ортогональные, а они, как известно из линейной алгебры, сохраняют скалярное произведение. В частности, матрица $V$ - обратима, а это значит, что множества векторов $\{x,\; x \neq 0\}$ и $\{Vx | \; Vx \neq 0$\}, совпадают. Тогда можем выписать такую цепочку:

$||A||^2_2 = \sup\limits_{x \neq 0} \frac{(Ax, Ax)}{(x, x)} = 
 \sup\limits_{Vx \neq 0} \frac{(UJV^TVx, UJV^TVx)}{(Vx, Vx)} =
 \sup\limits_{Vx \neq 0} \frac{(Jx, Jx)}{(x, x)} =
 \sup\limits_{x \neq 0} \frac{(Jx, Jx)}{(x, x)}$
 
В последнем равенстве мы пользовались, опять же, невырожденностью $V$, следующей из обратимости. Возьмем $||x|| = 1$, как и в предыдущих пунктах. Мы видим, что оптимальный для нас вектор $x: x_j = 1, x_i = 0, i \neq j $, где $ j = arg(\max\limits_{i = 1, .., n}(|\sigma_i|))$, и на нем достигается масимальное значение $(\max\limits_{i = 1, .., n}(|\sigma_i|))^2$. Это - ничто иное, как максимальное сингулярное число. В то же время для симметричной матрицы это модуль максимального собственного значения. Норма $L_2$ матрицы - её максимальное сингуляное число.

Что касается Фробениусовой нормы, она является средним квадратическим всех элементов матрицы, и очевидна аналогия с $l_2$.

\section{Вычисли это, ч. 1} 

 $\; \; \; \; \;$ Для начала важно упомянуть, что вся теория, описанная здесь(если не обговорено иное), относится прежде всего к комплекснозначным матрицам. 

$QR$-разложение - это представление квадратной матрицы  $ n\times n$: $A = QR$, где  $Q$ — унитарная матрица размера $ n\times n$, а  $R$ — верхнетреугольная матрица размера $ n\times n$. В интересующем нас случае, когда $A$ - действительнозначная матрица, $Q$ - ортогональная матрица. Разложение можно рассматривать как побочный продукт выполнения алгоритма Грамма-Шмидта, выполняемого за $O(n^3)$. Далее, система $Ax = b$ представляется после домножения на $Q^T$ слева в виде $Q^TQRx = Rx =Q^Tb$ (домножение корректно из-за того, что $Q$ - невырожденная матрица). Далее, задача очень сильно упрощена. т. к. $R$ - верхнетреугольная матрица, и можно решить систему методом прямого хода за квадрат. Итак, c учетом того, что и умножать матрицы друг на друга не требуется, итоговая сложность - $ O(n^3)$.

Добавление насчет произвольной размерности матрицы. $QR$, как уже говорилось, есть результат процесса Грамма-Шмидта, то есть вычитания из векторов, составлявших матрицу, предыдущих обработанных, с такими коэффициентами, чтобы они составляли ортогональный базис, и коэффициенты вычисляются через скалярные произведения этих в-ров. Очевидно, что данный процесс может быть окончен раньше, если кол-во векторов меньше, чем их размерность, и наоборот. Отсюда сложность - $O(nm^2)$ (соответственно, такова и итоговая сложность), $n$ - длина вектора, $m$ - их колличество. И $Q$ имеет размерность $m\times m$, а $R$ и $A$ - $m \times n$.

 $SVD$ - разложение матрицы $A$ размерности $m \times n$ - это сингулярное разложение, описанное в предыдущем пункте (рассматриваются действительнозначные матрицы, в комплекснозначном случае транспонирование заменяется на сопряженное транспонирование): $A = UJV^T$, $U$ и $V$ - ортогональные (унитарные в комплекснозначном случае) матрицы, $J$ - диагональная. Также, если разложение имеет данный вид (а $J = diag(\sigma_i) $), то решение уравнения $Ax = y$ имеет вид $Vdiag(1/\sigma_i)U^Ty$ (проверяется чисто подстановкой и использованием определения ортогональности, ну, ещё и перемножением диагональных матриц). Что же касается сложности самого разложения, она составляет $O(m^2n + n^3)$, при простом итеррационном методе. Этот метод есть результат поиска сингулярных векторов и значений - некоторой аналогии собственных векторов и значений(точнее, берутся собственные вектора матриц $AA^T$ и $A^TA$), отсюда и можно примерно понять сложность. Общая сложность - $ O(m^2n + n^3)$, так как матрицы перемножаются по-любому быстрее, чем выполняется разложение. 
 
\section{Вычисли это, ч. 2}
 
 $\; \; \; \; \;$Разреженная матрица - матрица, состоящая из преимущественно нулей (строгого определения нет, часто принимают колличество нулей - O(n)). Стандартное хранение её в качестве двумерного массива никаких очевидных выгод не дает. Также распространено хранение построчное хранение матриц - для строки хранятся пары из номера столбца с ненулевым элементом и его значения. Имеет логарифмическое время обращения к элементу посредством бинпоиска (при упорядочении элементов по столбцу в строке, но об этом после). Основной пример - форматы RR(C)O и RR(C)U. Оба заменяют двумерный массив на 3 одномерных, и при этом в одном хранятся все ненулевые значения, во втором - номера их столбцов, третий же отвечает за строки, в которых хранятся элементы (в массиве хранитятся номера элементов в первом массиве, с которых начинается очередная строка). Это -  вариации построчечного хранения, и RR(C)U отличается от RR(C)O тем, что элементы в строке могут храниться неупорядоченно: это полезно тогда, когда в процессе вычисления сложно задать порядок строк в результате сразу, но приходится перебирать всю строку, если хочется обратиться к определенному элементу по индексу. Возможен ещё вариант хранить массивы тройками "индексы - значение", но о применении их на практике автору этих строк ничего не известно.

\section{Диаграмма Вороного}

$\; \; \; \; \;$1.  Уравнение $||x - x_0||_2 \leq ||x - x_i||_2$ переписывается в виде $(x - x_0)(x - x_0) \leq (x - x_i)(x - x_i)$, что путем упрощения приводится к виду $ 2(x_i - x_0, x) \leq x_i^2 - x_0^2$, а это значит, что требуемая система - это $Ax \leq y$, i - я строка A -  транспонированный вектор  $2(x_i - x_0)$, и $y_i = x_i^2 - x_0^2$. Очевидно, что фигура может быть, скажем, неограниченной, а значит, многоугольник имеем в широком смысле этого слова - т. е. как раз фигура, заданная уравнением $Ax \leq y$. Ну и нетрудно догадаться, что его стороны лежат на серединных перпендикулярах к отрезкам, соединяющим $x_0$ и $x_i$. Разумеется, при $n > 2$ стороны и серединные перпендикуляры заменяются на очевидные многомерные аналоги.

2.  Из утверждения предыдущего пункта о серединных перпендикулярах можно сразу прийти к выводу, что нам требуется только "отзеркалить"(провести симметрию относительно грани) имеющуюся внутреннюю точку от сторон многоугольника, если она дана. Но с другой стороны, решение задачи неоднозначное: вполне могут иметься точки, никак не влияющие на диаграмму, если находятся достаточно далеко. Сама же точка $x$ нам нужна по-любому: достаточно рассмотреть случай, когда есть только точки $x$ и $x_1$, и вся диаграма вырождается в полуплоскость (все в двумерном пространстве). Очевидно, мы не можем сказать в этом случае практически ничего. Как мы ни возьмем $x$ в ней, $x_1$ можно будет получить симметрией относительно разделяющей их прямой. Мы, на самом деле, можем брать в качестве $x$ все такие точки, из которых можно опустить перпендикуляры на все стороны многоугольника. Но он выпуклый, так как задается матричным неравенством (фактически, пересечение полуплоскостей), а значит, из школьной геометрии, из любой его внутренней точки можно опустить перпендикуляры на все грани.

3.  Диаграмма Вороного для n точек - это разбиение плоскости на n многоугольников, $i$-й является диаграммой Вороного для $i$-й точки относительно всех остальных. Очевидно, что, для каждой точки пространства есть точка из данного множества, лежащая к ней как минимум не дальше всех остальных, т. е. диаграмма Вороного - действительно разбиение пространства (остаются только границы). Как было указано, в случае двух точек решение очень неоднозначное: множества диаграммы разделены прямой, и она будет диаграммой Вороного для любой пары точек, симметричной относительно этой прямой. Есть и более сложные примеры: плоскость может быть разделена на некоторое колличество квадратов в виде сетки(стороны квадратов продолжаются до бесконечности там, где сетка кончается, и мы в двумерном пространстве). Мы можем выбрать совершенно случайно одну точку внутри квадрата, а все остальные получать симметриями относительно его сторон. Как видим, семейства множеств, которые могут получиться в итоге, довольно обширны. Но в то же время нам достаточно взять всего лишь одну точку, чтобы получить все остальные(это доказывается простой индукцией). Но в то же время мы не може брать произвольную точку, т. к. симметричная ей может не лежать внутри соседнего многоугольника. 

4.  Диаграмма Вороного широко применяется в разделе computer science, связаненом с распознаванием и моделировании. Есть три эквивалентые (сводящиеся друг к другу) задачи: диаграмма Вороного, триангуляция Делоне и объемная выпуклая оболочка. Как сетки, так и скелеты трехмерных моделей строятся с использованием триангуляции Делоне и диаграммы Вороного. Построение 3d-выпуклой оболочки - одна из базовых функций трехмерных редакторов. Роботы часто ищут свой путь при помощи триангуляции Делоне. Диаграммы Вороного применяются в распозновании объектов на фото/видео.

\section{Выпуклый или конический}
\begin{itemize}  
\item $ \{ x \in \mathbb{R}^n \;|\; \alpha \leq a^Tx \leq \beta \} $. То, что множество не коническое, очевидно из того, что $x$ можно умножить на любое, сколь угодно большое (и неотрицательное) число $\theta$, увеличивается в соответствующее число раз и скалярное произведение. Будет либо $a^T(\theta x) > \beta $, либо $a^T(\theta x) < \alpha$, в зависимости от изначального знака скалярного произведенияя. В то же время пусть $ \alpha \leq a^Tx \leq \beta$, $ \alpha \leq a^Ty \leq \beta$. Тогда при $t \in [0, 1]$
$\alpha \leq t\alpha + (1 - t)\alpha \leq t(a^Tx) + (1 - t)(b^Ty) = (a^T(tx + (1-t)y) = t(a^Tx) + (1 - t)(b^Ty) \leq t\beta + (1 - t)\beta \leq \beta$. Множество - выпуклое.
\item $\{x \in \mathbb{R}^n \;| \; a_1^Tx \leq b_1, a^T_2x \leq b_2\}$. Множество не является конусом ровно по тем же соображениям, что и в предыдущем пункте, если $b_1 > 0$ или $b_2 > 0$. Но если $b_1 < 0$ или $b_2 < 0$, мы можем поделить $x$ на сколь угодно малое число и получить нарушение условия. Наконец, случай $b_1 = 0, b_2 = 0$, очевидно, задает конус. С другой стороны, рассматривая выпуклость, в предыдущем номере без ограничения логики можно взять $\alpha = -\infty$, избавившись от одного неравенства. А далее остается лишь воспользоваться доказанным на семинаре утверждением о замкнутости множества выпуклых множеств относительно пересечения.
\item $\{x \; |\; \; \|x - x_0\|_2 \leq \|x - y\|_2, \forall y \in S \subseteq R^n\}$
Неравенство из условия можно переписать в виде 
$(x - x_0)(x - x_0) \leq (x - y)(x - y)$, или
$ 2(y - x_0, x) \leq (y, y) - (x_0 - x_0)$.
А это уже рассмотренное нами уравнение вида $a^Tx \leq \beta$. Домножением $x$ на константу можно нарушить условие, а значит, множество не коническое. Осталось заметить что доказанное нами утверждение о замкнутости относительно пересечения не только по конечному множеству, но и когда, скажем, $y$ "пробегает" $S$. Таким образом, множество выпукло.
\item  $\{x \;| \; \|x - a\|2 \leq \|x - b\|_2\}, a \neq b$. Пусть некоторый вектор $x$ принадлежит данному множеству. Рассмотрим вектор $lx$:  $||lx - a||^2 \leq \theta||lx - b||^2$ Приведем неравенство к следующему виду: $l^2(1 - \theta^2)(x, x) + 2l(\theta^2 b - a)x + a^2 - \theta^2 b^2 \leq 0$. При $l \rightarrow +\infty, \theta < 1$ левое выражиние стремится к $+\infty$, неравенство не выполняется, точка $lx$ множеству не принадлежит, это не конус. При $\theta = 1$ задача сводится к предыдущей.

 Далее, покажем, что наше множество, за исключением вырожденных случаев, является сферой. Довольно очевидно(проверяется по определению), что выпуклость множества сохраняется при сдвиге. Сделаем замену  $x \rightarrow x - a$, $b - a = c$. Уравнение примет вид $||x||_2 \leq \theta||x - c||_2$. Далее, распишем это неравенство, возведенное в квадрат:
  $(x, x) \leq \theta^2((x, x) - 2(x, c) + (c, c))$ \\
  $(1 - \theta^2)(x, x) + 2\theta^2(x, c) + \theta^2(c, c) \leq 0\\$
Случай $\theta = 0$ нам не интересен, т. к. тогда множество - точка, считающаяся выпуклым множеством. Случай $\theta = 1$ соответствует(см. посл. ур-е) полупространству, тоже выпуклому множеству. Остается $\theta \in (0, 1)$ , можем делить:  \\
$\displaystyle (x, x) + 2\frac{\theta^2}{1 - \theta^2}(x, c) - \frac{\theta^2}{1 - \theta^2}(c, c) \leq 0$\\
Сделаем замену $\displaystyle c = \frac{\theta^2}{1 - \theta^2}$. Тогда получим: $(x, x) + 2(x, u) + (u, u) = (x + u, x + u) \leq r^2$, и $r$ - некоторая константа, котораяя нам не важна, даже если она меньше нуля: пустое множество считаем выпуклым. Снова делаем параллельный перенос: $(x, x) \leq r^2$. Докажем, что шар - выпуклое множество: при $(x, x) \leq r^2, (y, y) \leq r^2$ имеем:\\
$(tx + (1 - t)y)^2 = t^2(x, x)+ (1 - t)^2(y, y) + 2t(1 - t)(x, y) \leq
t^2r^2+ (1 - t)^2r^2 + 2t(1 - t)r^2 = (t + 1 - t)^2r^2 = r^2$\\
Здесь мы воспользовались тем, что скалярное произведение двух векторов не может быть больше квадрата модуля одного из них. Видим, что множество выпуклое.
\end{itemize}   
\end{document}
 