\documentclass{article}
\usepackage[utf8]{inputenc}

\title{Домашнее задание № 5}
\author{Иван Нечепуренко }
\date{September 2018}

\usepackage{natbib}
\usepackage{graphicx}
\usepackage{amsmath,amsfonts,amssymb,amsthm,mathtools} % AMS
\usepackage[english,russian]{babel}	% локализация и переносы

%%% Работа с картинками
\usepackage{graphicx}  % Для вставки рисунков
\graphicspath{{images/}}  % папки с картинками
\setlength\fboxsep{3pt} % Отступ рамки \fbox{} от рисунка
\setlength\fboxrule{1pt} % Толщина линий рамки \fbox{}
\usepackage{wrapfig} % Обтекание рисунков и таблиц текстом

\begin{document}
Заметим, что из невырожденности $A$ следует:

$$\sum\limits_i \alpha_ip_i = 0 \Leftrightarrow A(\sum\limits_i \alpha_ip_i  ) = 0
\Leftrightarrow 
\sum\limits_i \alpha_iAp_i = 0 $$
Прямо отсюда следует, что условия линейной независимости векторов $\{Ap_i\}$ и $\{p_i\}$ эквивалентны.
Теперь предположим, что 
$$ \sum\limits_i\alpha_iAp_i = 0, \exists \alpha_i \neq 0$$
Домножим выражение слева на $p_i^T$, получим
$$ \alpha_ip_i^TAp_i = 0$$
$$ p_i^TAp_i = 0$$
Но тогда, так как $A \in \mathbb{S}^n_{++}$, $p_i = 0$, а это проворечит условию. Но тогда
$\{Ap_i\}$, и соответственно, $\{p_i\}$, линейно независимы.

Задача 1
Проверьте, что направления в методе сопряжённых градиентов для квадратичной целевой функции и в методе Флетчера-Ривса являются направлениями убывания. Для любой ли стратегии линейного поиска шага в методе Флетчера-Ривса полученное направление будет направлением убывания? Почему?

Даёт ли процедура дробления шага шаг, удовлестворябщий условию Вольфа? Если нет, то почему и как её нужно модифицировать, чтобы найти шаг, удовлетворяющий условию Вольфа?

В доказательстве градиентного спуска для квадратичной функции опираемся на утверждения и определения из лекционной презентации. Расписываем выражение $f(x_{k + 1}) - f(x_k)$ в лоб:
$$f(x_{k + 1}) - f(x_k) = \frac{x_{k + 1}Ax_{k + 1} - x_kAx_k}{2} - b^T(x_{k + 1} - x_k) = 
\{ x_{k + 1} = x_k + \alpha_k x_k \} = $$
$$ = \alpha_k x_k^T A p_k + \frac{\alpha_k^2 p_k^TAp_k}{2} - \alpha_k b^T p_k = 
\alpha_k ((x_k^T A - b^T) p_k + \alpha_k \frac{p_k^TAp_k}{2} )
$$
Теперь пользуемся тем, что $Ax_k - b = r_k$, $ \displaystyle \alpha_k = \frac{r_k^T r_k}{p_k^T A p_k} $
$$f(x_{k + 1}) - f(x_k) = \alpha_k(r_k^Tp_k + \frac{r^T_kr_k}{2} )
= \alpha_k(-r^T_k r_k + \frac{r^T_kr_k}{2} )  =  - \alpha_k \frac{r^T_kr_k}{2}  \leq 0$$
При этом равенство будет достигаться только при $r_k = 0$, т. е. в целевой точке.
\end{document}